\documentclass{article}
\usepackage{fancyhdr}

\pagestyle{fancy}
\lhead{}
\chead{Classes, Classification and Meta-classes in XMF}
\rhead{\thepage}
\cfoot{(c) 2004 Ceteva Ltd.}

\title{Classes, Classification and Meta-classes in XMF\\(Incomplete Draft)}

\author{Ceteva Ltd.}

\begin{document}

\maketitle

\section{Introduction}

XMF is a class-based object-oriented modelling environment. Each value in XMF has a type or {\em classifier}
that describes its structure and behaviour. Values in XMF are divided into objects and non-objects. Object
types are called {\em classes} and non-object types are called {\em classifiers}. If a value {\em v} is of a 
type {\em c} then we say that {\em v} is an instance of {\em c}. XMF is provided with a large number of classes 
and classifiers; XMF developers can define their own classes and classifiers as extensions of those provided.

Classes and classifiers {\em classify} their instances by running {\em constraints}. A constraint is a
boolean valued expression that runs in the context of the current state of the candidate instance. The
outcome of constraint checking is a constraint report containing details of the constraints that were
performed, the candidates, the outcome and a reason for any constraints that failed. Constraint checking is
a powerful mechanism for checking whether a model or a model scenario is correctly formed.

XMF is a meta-modelling environment. Values are instances of classes and classifiers. Classes and classifiers 
are themselves instances of meta-classes; the meta-classes describe the structure and behaviour of their
instances - including information relating to constraint checking. New types of classes can be created by
extending the basic meta-classes provided by XMF. Meta-extensions are a powerful tool that allows XMF to be
extended with new modelling features.

This note describes features of XMF classes and classifiers. 

\section{Classes}

A class describes the structure and behaviour of its instances. A class has a name and lives in a name-space.
The following is a basic class that lives in the name-space {\tt Root}:
\begin{verbatim}
context Root
  @Class EmptyClass end
\end{verbatim}
By default, the class {\tt EmptyClass} specializes the XMF class {\tt Object} and provides a single
constructor for creating instances: {\tt EmptyClass()}. If we perform the following expression:
\begin{verbatim}
EmptyClass().isKindOf(EmptyClass)
\end{verbatim}
then the result is {\tt true} since the newly created instance is directly an instance of {\tt EmptyClasss}.
In addition, the expression:
\begin{verbatim}
EmptyClass().isKindOf(Object) and EmptyClass().isKindOf(Element)
\end{verbatim}
returns true since {\tt EmptyClass} inherits (bu default) from {\tt Object} and {\tt Object} inherits
from {\tt Element} (the class {\tt Element} does not inherit from anywhere). The class {\tt EmptyClass}
is itself a value:
\begin{verbatim}
EmptyClass.isKindOf(Class) and 
EmptyClass.isKindOf(Classifier) and 
EmptyClass.isKindOf(NamedElement) and
EmptyClass.isKindOf(Object) and 
EmptyClass.isKindOf(Element)
\end{verbatim}
returns {\tt true}.

\subsection{Attributes}

A class typically defines some attributes that correspond to slots in the instancs of the class.
Each attribute has a name and a type and may optionally have some modifiers and an initial value.
The following is a simple example of a class with attributes:
\begin{verbatim}
context Root
  @Class Point 
    @Attribute x : Integer end
    @Attribute y : Integer end
    @Constructor(x,y) ! end
  end
\end{verbatim}
A new point is creates using the constructor defined by {\tt Point}. A class may define any number of
constructors; each constructor must have a different number of arguments. Each constructor argument
corresponds to one of the attributes in the class. The optional modifier {\tt !} declares that the
printed representation for a point is defined by that constructor.
A new {\tt Point} is constructed:
\begin{verbatim}
Point(100,200)
\end{verbatim}
where {\tt 100} is the value of the slot {\tt x} and {\tt 200} is the value of the slot {\tt y}.
Accessor and updater operations are automatically produced by including attribute modifiers:
\begin{verbatim}
context Root
  @Class Point 
    @Attribute x : Integer (?,!) end
    @Attribute y : Integer (?,!) end
    @Constructor(x,y) ! end
  end
\end{verbatim}
The modifier {\tt ?} defines that operations {\tt getX} and {\tt getY} are automatically provided;
modifier {\tt !} defines that operations {\tt setX} and {\tt setY} are automatically provided:
\begin{verbatim}
let p = Point(100,200)
in p.setX(p.getX() - 1);
   p.setY(p.getY() - 1)
end
\end{verbatim}

\subsection{Inheritance}

All classes are specializations or {\em sub-classes} of at least one other class. By default
a class is a sub-class of {\tt Object}. A class inherits constructors, attributes, operations
and constraints from its parent. For example the following class specializes {\tt Point}
with a {\tt z} co-ordinate:
\begin{verbatim}
context Root
  @Class Point3D extends Point
    @Attribute z : Integer (?,!) end
    @Constructor(x,y,z) ! end
  end
\end{verbatim}
An instance of {\tt Point3D} may be constructed using a two-place constructor or a three-place 
constructor (the three-place is probably more useful). An instance of the class {\tt Point3D} is
also an instance of the parent class {\tt Point}:
\begin{verbatim}
let p = Point3D(1,2,3)
in p.isKindOf(Point)
end
\end{verbatim}
returns {\tt true}.

\subsection{Operations}

Object-oriented execution proceeds by message passing; a message consists of a name and
some argument values. When a message is sent to an object, the name is looked up in the class
of the object (and its parents); if an operation is found then it is invoked otherwise an
error is reported. Operations may be defined as part of a class or a package definition or 
added to existing classes and packages via a {\tt context} defintion.

The following defines a class of stacks and adds the definition to a package named {\tt Stacks}.
The example shows the complete contents of a file containing the definition. XMF can be used
in a file-based mode where files contain source code that is compiled using the XMF compiler.
The compiled binary is then loaded into XMF.
\begin{verbatim}
parserImport XOCL;

context Stacks

  @Class Stack
  
    @Attribute elements : Seq(Element) end
    @Attribute index : Integer end 
    
    @Operation isEmpty():Boolean
      self.elements->isEmpty
    end
    
    @Operation pop()
      if self.isEmpty()
      then throw StackUnderflow(self)
      else 
        let head = elements->head
        in self.elements := elements->tail;
           head
        end
      end
    end
    
    @Operation push(e:Element)
      self.elements := Seq{e | elements}
    end
    
   @Operation top()
      if self.isEmpty()
      then throw StackUnderflow(self)
      else elements->head
      end
    end
    
  end
\end{verbatim}  

\section{Constraints}

\section{Classifiers}

\section{Meta-classes}

\end{document}